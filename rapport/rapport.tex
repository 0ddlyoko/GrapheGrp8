\documentclass{beamer}

\usepackage[utf8]{inputenc} % Permet de tapper les accents tels quels
\usepackage[T1]{fontenc} % Permet l'utilisation d'accents
\usepackage[french]{babel} % Document en français

\usetheme{Warsaw} %theme utiliser
 
\begin{document}
 
\title{Projet de théorie des graphes et optimisation combinatoire:}
\subtitle[\ldots]{Détection de communauté dans les graphes}
\author{Coyez François, Giocomello Nathan, Kamta Boris}
\institute{Université de MONS\\Faculté des Sciences\\BAC 3 Science - Informatique}
\date{Année académique 2019-2020\\sous la coordination de:\\ D.Tuyttens}
\maketitle
 
 \begin{frame} % debut slide
\frametitle{Pr\'esentation du projet} %titre du slide
\framesubtitle{Probl\`eme:} %sous-titre du slide
La recherche de sous-ensemble de nœuds d'un r\'eseau qui sont fortement connect\'es( c’est-\`a-dire ayant entre eux un nombre important de connexions qu’avec le reste du r\'eseau) en utilisant l'Optimisation de la modularit\'e.
\end{frame} % fin slide

\begin{frame} % debut slide
\frametitle{Présentation du projet} %titre du slide
\framesubtitle{Objectifs:} %sous-titre du slide
Il est question pour nous, \`a partir de fichiers fournis par l'enseignant (fichiers mod\'elisant un graphe par son nombre de sommets, nombre d'arr\^etes, le tableau des listes de successeurs et le tableau des têtes de listes), de produire un programme basé sur les M\'etaheuristiques vues en cour, prenant en entrée l'un des fichiers et fournissant en sortie le calcul de la valeur de la modularit\'e, le nombre de communaut\'es, le nombre de sommet(s) par communaut\'e et la liste de sommet(s) de ses communaut\'es.
\end{frame} % fin slide

\begin{frame} % debut slide
\frametitle{Présentation du projet} %titre du slide
\framesubtitle{Choix personnels: outils} %sous-titre du slide
Pour r\'ealiser notre projet, nous avons decidé de choisir comme language de programmation Java.
\end{frame} % fin slide

\begin{frame} % debut slide
\frametitle{Présentation du projet} %titre du slide
\framesubtitle{Choix personnels: m\'ethode} %sous-titre du slide

\end{frame} % fin slide
 

\begin{frame} % debut slide
\frametitle{L’algorithme} %titre du slide

\end{frame} % fin slide


\begin{frame} % debut slide
\frametitle{Les jeux de tests} %titre du slide

\end{frame} % fin slide


 

\end{document}